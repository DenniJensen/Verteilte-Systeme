\section{Snapshot Algorithmus}
\label{sec:snapshotalg}

- Annahmen für den Algorithmus
  - Kanal zwischen 2 Knoten haben unendlichen Speicher, sind Fehlerfrei,
    die Nachrichten kommen in der Reihenfolge an, wie sie gesendet werden 
    (FIFO Prinzip, First in, First out)
    
- Beim Start ein Knoten
  - speichert seinen Aktuellen Zustand
  - sendet Marker Nachricht zu allen seinen bekannten Nachbarn
- eventuell startknoten per Leaderelection gewählt.

- danach wie in Listing \ref{lst:markerarrive} dargestellt, wenn Marker Nachricht ein Knoten erreicht

- jeder Knoten (start Knoten oder wenn erste Marker Nachricht erreicht.
  - speichern des aktuellen Zustand
  - sendet Marker Nachricht zu allen Nachbarn
  - speichert Kanäle als Sequenz von Nachrichten bis eine Marker Nachricht
   über den Kanal erreicht 
  - wenn alle Kanäle gespeichert sind, dann ist der Knoten Fertig

- eventuell senden alle Knoten ihre Zustände mit den Sequenzen der Kanäle zu einem globalen Knoten
   
   
\begin{lstlisting}[caption={Pseudo Ablauf, wenn Marker Nachricht über den Kanal c den Knoten q erreicht. \cite{snapshotChandyLamport}}, label=lst:markerarrive]
if q hat seinen Zustand noch 
nicht gespeichert then
  q speichert seinen Zustand;
  q speichert c als leere Sequenz;
  q sendet an alle Nachbarn 
  eine Marker Nachricht;
else
  q speichert den Zustand von c 
  als Sequenz von Nachrichten
  von dem Zeitpunkt als q 
  seinen Zustand gespeichert 
  hat und dem Zeitpunkt als die
  Marker Nachricht erhaelt;
\end{lstlisting}
