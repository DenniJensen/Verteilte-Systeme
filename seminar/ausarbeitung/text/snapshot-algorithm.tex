\section{Snapshot Algorithmus}
\label{sec:snapshotalg}
Annahme für den Algorithmus ist, dass der Kanal zwischen zwei Knoten unendlich
Speicher zur Verfügung stehen. Es wird auch davon ausgegangen, dass die
Übertragung der Nachrichten fehlerfrei ist und die Reihenfolge der Nachrichten
nach dem FIFO-Prinzip abgearbeitet werden.

Ein gestarteter Knoten speichert seinen aktuellen Zustand und sendet eine
Marker-Nachricht an alle bekannte Nachbarknoten.  Der Startknoten kann per
Leaderelection gewählt werden.

Wie in Listing \ref{lst:markerarrive} dargestellt, speichert jeder Knoten den
aktuelle Zustand, wenn Marker-Nachricht eintrifft. Dann wird eine
Marker-Nachricht an alle bekannten Knoten versendet.  Alle Kanäle werden als
Sequenz von Nachrichten gespeichert bis eine Marker-Nachricht über den Kanal
erreicht wird. Abschlossen mit der Aufnahme ist der Knoten, wenn alle Kanäle
gespeichert sind.  Eine Möglichkeit besteht, dass alle Knoten ihren Zustand mit
den Sequenzen der Kanäle an einen globalen Knoten versenden.

\begin{lstlisting}[caption={Pseudo Ablauf, wenn Marker Nachricht über den Kanal c den Knoten q erreicht. \cite{snapshotChandyLamport}}, label=lst:markerarrive]
if q hat seinen Zustand noch
nicht gespeichert then
  q speichert seinen Zustand;
  q speichert c als leere Sequenz;
  q sendet an alle Nachbarn
  eine Marker Nachricht;
else
  q speichert den Zustand von c
  als Sequenz von Nachrichten
  von dem Zeitpunkt als q
  seinen Zustand gespeichert
  hat und dem Zeitpunkt als die
  Marker Nachricht erhaelt;
\end{lstlisting}
