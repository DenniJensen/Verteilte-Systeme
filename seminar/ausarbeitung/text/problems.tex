\section{Probleme}
\label{sec:probs}
In diesem Abschnitt berichten wir die Probleme bei der Implementierung des
Snapshot-Algorithmus mit Akka.

Ein Startproblem bei der Implementierung lieferte Akka.
Um ein Akka-Netzwerk zu starten, indem alle Knoten miteinander kommunizieren
können, benötigten wir einen Initial-Knoten zum Start des Netzwerkes.
Außerdem die Konfigurationsdatei, in welcher die Kommunikation definiert wird.
Damit die Knoten im Netzwerk erkannt werden, muss sich jeder Knoten beim
Initial-Knoten anmelden.

Bei der Implementierung des Snapshots wurde ein Problem mit der Zeit beim 
Starten des Snapshots erkannt.
Es musste in Timeout eingeführt werden, damit alle Knoten, beziehungsweise
genug Knoten, dem Netzwerk beigetreten sind bevor die ersten Marker-Nachrichten
gesendet wurde.
Sonst kam es dazu das die erste Marker-Nachricht vor dem Start eines Knoten an
den initialen Knoten gesendet wurde und somit kein Snapshot durchgeführt wurde.
