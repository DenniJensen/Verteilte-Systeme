\section{Akka}
\label{sec:akka}

Akka ist ein Toolkit und Runtime zum bauen von neben-läufigen, verteilten und
belastbaren Nachricht betriebene Applikationen in der JVM. \cite{akka}
Für die Implementierung wird der \textit{UntypedActor} von Akka erweitert.
In diesem wird die \textit{onReceive(Object message)} Methode implementiert, 
welche zur Bearbeitung der Nachrichten vom Akka System dient.
Als Nachrichten werden Objekte verwendet.
Für jede Nachricht wurde eine Klasse erstellt, wie zum Beispiel zum Joinen 
oder der Marker-Nachricht.

Zum Starten der Aktoren wird ein ActorSystem erstellt.
In diesem System werden die einzelnen Knoten gestartet oder jeder Knoten
erhält ein eigenes System.
Die Systeme behandeln die Nachrichten und schicken sie den jeweiligen Aktor.
Zur Konfiguration dient die \textit{application.conf} Datei, welche zum 
Beispiel den Port und die IP angibt.
In unseren Fall steht dort die IP des initialen Knoten zu welchen nach dem 
Start eingeloggt wird.
Die einzelnen Aktoren, werden über die \textit{ActorRef} des Aktor Systems
angesprochen.

Jedes Wallet ist in der Implementation ein eigener Aktor im System.
Nach dem alle Aktoren gestartet sind, wird über die \textit{tell} Funktion 
vom \textit{ActorRef} der Snapshot gestartet.
