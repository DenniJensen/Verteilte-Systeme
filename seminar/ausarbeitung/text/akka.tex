\section{Akka}
\label{sec:akka}

Akka ist ein Toolkit und Runtime zum bauen von neben-läufigen, verteilten und
belastbaren Nachricht betriebene Applikationen in der JVM. \cite{akka}
Für die Implementierung wird der \textit{UntypedActor} von Akka erweitert.
In diesem wird die \textit{onReceive(Object message)} Methode implementiert, 
welche zur Bearbeitung der Nachrichten vom Akka System dient.
Als Nachrichten werden Objekte verwendet.
Für jede Nachricht wurde eine Klasse erstellt, wie zum Beispiel eine Nachricht
zum Joinen des Netzwerkes oder die Marker-Nachricht.

Zum Starten der Aktoren wird ein ActorSystem erstellt.
In diesem System werden die einzelnen Knoten, beziehungsweise die Aktoren,
gestartet oder jeder Knoten im Netzwerk erhält ein eigenes System, wenn diese
nicht auf dem selben Rechner gestartet werden.
Die Systeme verwalten die reinkommenden und ausgehenden Nachrichten und 
schicken sie den jeweiligen Aktor.
Zur Konfiguration dient die \textit{application.conf} Datei, in welcher zum 
Beispiel der Port und die IP angegeben ist.
In unseren Fall steht dort die IP des initialen Knoten zu welchen nach dem 
Start eine Verbindung hergestellt werden soll um den Netzwerk beizutreten.
Die einzelnen Aktoren, werden dann über die \textit{ActorRef} des Aktor Systems
angesprochen.

Jedes Wallet ist in der Implementation ein eigener Aktor im System.
Nach dem alle Aktoren gestartet sind, wird über die \textit{tell} Funktion 
vom \textit{ActorRef} die erste Marker-Nachricht versendet und damit der 
Snapshot Algorithmus gestartet.
