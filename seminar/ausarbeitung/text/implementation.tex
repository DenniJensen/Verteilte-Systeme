\section{Implementation}
\label{sec:impl}
Für die Implementation des Snapshot-Alogrythmus wird eigentlich ein Leader 
notwendig, welcher benötigen wird um den Snapshot zu starten. Dies kann zum
Beispiel über einen Leader Election Algorithmus erfolgen. Da es unsere Aufgabe
war nur einen Algorithmus zu implentieren, wurde beschlossen, dass der 
Initialknoten im Akka-System der Leader für den Snapshot ist. Dieser Knoten 
erhählt nach dem Start die erste Marker-Nachricht um den Snapshot zu starten.

Alle Bestandteile des Snapshot sind in einer Klasse implementiert. Nach dem
Abschluss eines Snapshot eines Knotens wird dieses Objekt zum Initialknoten
geschickt.
Die Variablen und Methoden der Klassen sind in Anhang im Listing \ref{lst:sclass} 
dargestellt.

Folgende Variablen werden genutzt: als erstes der Name des Wallets, dann
der aktuelle Betrag, welcher beim Start der Aufnahme vom Knotens gehalten wird,
eine Liste seine Nachbarknoten, welche noch keine Marker-Nachricht gesendet
haben und eine Hash-Map bestehend aus Knoten und deren Nachrichten vom Start
der Aufnahme bis zum erhalt der Marker Nachricht von diesem Knoten.

\begin{itemize}
 \item \texttt{name: } Name des Wallets
 \item \texttt{amount: } Betrag beim Start der Aufnahme
 \item \texttt{waitingToResponse: } Liste der Knoten, welche noch keine 
 Marker-Nachricht gesendet haben.
 \item \texttt{recordedMessages: } HashMap aus den Knoten und allen Nachrichten, 
 welche von diesen ankommen, seid dem Start der Aufnahme und dem erreichen der
 Marker-Nachricht
\end{itemize}

Ein Snapshot startet mit dem Namen des Knotens, dessen aktuellen Betrag und 
einer Liste von Knoten, von denen auf Marker-Nachrichten gewartet wird.
Bei Aufnahme werden alle Nachrichten von der \textit{handleMessages} Methode
verarbeitet und in der Hash-Map aufgenommen.
Ist die Nachricht eine Marker-Nachricht wir der Kanal geschlossen und der
Knoten aus der waitingToResponse-Liste gelöscht.
Wenn die waitingtoResponse-Liste leer ist, dann ist der Snapshot am Knoten
abgeschlossen und das Objekt wird zum Initialknoten gesendet.

Die \textit{handleMarkerMessage} Methode des Wallets, im Anhang in Listing 
\ref{lst:makerHandler} dargestellt, kommt zum Einsatz wenn eine Marker-Nachricht
beim Knoten eintrifft.

Ist der Snapshot nocht nicht im Aufnahme-Modus, erstellt er eine Liste der
bekannten Knoten, dann einen neuen Snapshot mit seinem Namen, dem aktuellen
Betrag und der Liste. Anschließen werden Marker-Nachrichten an alle Nachbarn
verschickt.
Sollte die Marker-Nachricht vom einem Knoten kommen, welche dem Knoten nicht
bekannt ist, wird diesem eine Marker-Nachricht zurück geschickt.
Dies wird getan um sicher zu stellen, das der Algorithmus auch beendet wird.

Am Schluss, wenn alle Knoten ihren Snapshot zum Initialknoten gesendet haben,
lassen wir die Snapshots in der Konsole ausgegeben.
