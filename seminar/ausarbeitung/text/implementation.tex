\section{Implementation}
\label{sec:impl}
Um den Snapshot-Alogrythmus zu implementieren, benötigen wir für den
Algorythmus einen Leader. In unserer Implemeniterung ist der Initialknoten im
Akka-System der Leader für den Snapshot. Dieser Knoten erhählt nach dem Start
die erste Marker-Nachricht.

Alle Bestandteile des Snapshot sind in einer Klasse implementiert. Im Laufe
des Programms wird diese Klasse über das Sysytem verschickt.
Variablen und Methoden der Klassen sind in \ref{lst:sclass} dargestellt.

Folgende Variablen werden genutzt: als erstes der name des Wallets, dann
der Betrag beim Start der Aufnahme des Knotens in das System, eine Liste der
Knoten und allen Nachrichten von Knoten, die sich in der Liste befinden und
eine Hash-Map bestehend aus Knoten und deren Nachrichten.

\begin{itemize}
 \item name = name des Wallets
 \item amount = Betrag beim start der Aufnahme
 \item waitingToResponse = Liste der Knoten, welche noch keine Marker Messsage
  gesendet haben
 \item recordedMessages = HashMap aus den Knoten und allen Nachrichten, welche
  von diesen ankommen, seid dem Start der Aufnahme und dem erreichen der
   Marker Nachricht
\end{itemize}

Ein Snapshot startet mit dem Namen des Knotens, dessen Betra und einer Liste
von Kanälen, die auf Marker Nachrichten warten.
Bei Aufnahme werden alle Nachrichten von der \textit{handeMessages}
verarbeitet und in der Hash-Map aufgenommen.
Ist die Nachricht eine Marker-Nachricht wir der Kanal geschlossen und der
Knoten aus der waitingToResponse-Liste gelöscht.
Die \textit{handleMarkerMessage} kommt zum Einsatz wenn eine Marker-Nachricht
beim Knoten eintrifft.

Ist der Snapshot nocht nicht im Aufnahme-Modus, erstellt er eine Liste der
bekannten Knoten, dann einen neuen Snapshot mit einem Namen, dem aktuellen
Betrag und der Liste. Anschließen werden Marker-Nachrichten an alle Nachbarn
verschickt
Ist der Knoten in dem Fall, von dem eine Marker-Nachricht eintritt, wird
an diesem eine extra Marker-Nachricht versendet um die Wartezeit des
Knotens zu verringern.

Sind alle Marker-Nachrichten aufgenommen, wird der Snapshot zum Initial-Knoten
versendet.
