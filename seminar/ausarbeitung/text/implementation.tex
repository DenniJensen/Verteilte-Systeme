\section{Implementation}
\label{sec:impl}

- wir haben einen Initialknoten, welcher bei uns der Leader ist.
- nach dem start, senden wir diesem die erste Marker Nachricht

- Der Snapshot ist als eigene Klasse implementiert, um ihn später 
  verschicken zu können
- in Listing \ref{lst:sclass} variablen und Methoden dargestellt

Variablen:
 - name = name des Wallets
 - amount = Betrag beim start der Aufnahme
 - waitingToResponse = Liste der Knoten, welche noch keine Marker Messsage
  gesendet haben
 - recordedMessages = HashMap aus den Knoten und allen Nachrichten, welche
  von diesen ankommen, seid dem Start der Aufnahme und dem erreichen der
   Marker Nachricht
   
- Snapshot startet mit
  - namen des Knoten, Betrag des Knoten beim start und Liste von Kanälen, 
  von denen auf eine Marker Nachricht gewartet wird
  
- solange aufgenommen wird, wird jede Nachricht an \textit{handeMessages} 
  übergeben um in die Hashmap aufgenommen zu werden
  - wenn es eine Makermessage ist, wird der Kanal geschlossen, aus
   waitingToResponse gelöscht
   
- Wenn Maker Message erreicht wird im Wallet \textit{handleMarkerMessage}
 aufgerufen
 
- wenn der Snapshot noch nicht aufgenommen wurde
  - erstelle Liste mit den bekannten Knoten
  - erstelle neuen Snapshot mit Name, aktuellen Betrag und der Liste
  - danach sende an alle Nachbarn eine Marker Nachricht
- Wenn er der Knoten unbekannt ist, von dem die Marker Nachricht kommt, 
  wird diesem noch extra eine Marker Nachricht gesendet um sicher zu 
  stellen, dass er nicht warten muss
  
- sollten alle Kanäle aufgenommen sein, wird der Snapshot noch zum initial 
  Knoten gesendet